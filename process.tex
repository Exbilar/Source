\documentclass[landscape]{article}
\usepackage{ctex}
\usepackage{amsmath}
\usepackage{amsfonts}
\usepackage{amssymb}
\usepackage{graphicx}
\usepackage{colortbl}
\usepackage{fancyvrb}
\usepackage{longtable}
\usepackage{xcolor}
\usepackage[hidelinks]{hyperref}
\usepackage[affil-it]{authblk}
\usepackage[top = 1.0in, bottom = 1.0in, left = 1.0in, right = 1.0in]{geometry}
\usepackage{amsthm}

\newcommand\spc{\vspace{6pt}}
\newcommand{\floor}[1]{\lfloor {#1} \rfloor}
\newcommand{\ceil}[1]{\lceil {#1} \rceil}
\newcommand*\chem[1]{\ensuremath{\mathrm{#1}}}

\newtheorem{theorem}{Theorem}[section]
\newtheorem{lemma}[theorem]{Lemma}

\date{Latest Update : \today}
\title{$\mathcal Process$}
\author{$\mathcal Exbilar$}

\begin{document}

\maketitle

\begin{longtable}{ccccccccccc}
  
  \hline
  Date & Name & Source & Status & Algorithm\\
  \hline
  2017.10.3 & clique & test & AC & 把绝对值化掉,树状数组优化DP\\
  \hline
  2017.10.3 & mod & test & AC & 势能线段树,区间取模裸题\\
  \hline
  2017.10.3 & number & test & AC & 恶心的数位DP\\
  \hline
  2017.10.4 & mine & test & AC & 考虑多种情况,考虑第$i$位以及前面两位的情况,恶心的DP\\
  \hline
  2017.10.4 & gcd & test & AC & 用莫比乌斯函数容斥就可以了,考试的时候被卡常了\\
  \hline
  2017.10.4 & water & test & AC & floodfill裸题,不过没学,当时不会\\
  \hline
  2017.10.6 & string & test & AC & 开$26$棵线段树,考场被卡常了\\
  \hline
  2017.10.6 & matrix & test & AC & 按列DP,考虑前$i$列$j$列在右区间的方案数\\
  \hline
  2017.10.6 & big & test & 理性AC & 字典树上DP\\
  \hline
  2017.10.7 & kurisu & test & AC & 数学+树状数组,考场上取模挂了\\
  \hline
  2017.10.7 & mayuri & test & AC & 约瑟夫问题经典优化\\
  \hline
  2017.10.7 & okarin & test & AC & DP+单调性优化,可以在DP转移的过程中保存max\\
  \hline
  2017.10.9 & matrix & test & AC & 树状数组+二维数点\\
  \hline
  2017.10.9 & present & test & AC & 剩余系+最短路,之前没学\\
  \hline
  2017.10.9 & Mahjong & test & AC & 恶心大模拟\\
  \hline
  2017.10.10 & Adore & test & AC & 状压DP,对每个点方案数的奇偶性DP\\
  \hline
  2017.10.10 & Confess & test & AC & 暴力\\
  \hline
  2017.10.10 & Repulsed & test & AC & 自底向上贪心,用DP维护(未)分配点的距离\\
  \hline
  2017.10.11 & Flowers & CF451E & AC & 容斥原理+组合计数,状压超出的物品\\
  \hline
  2017.10.11 & Permutations & CF396D & AC & 类似数位DP的做法,计算大小为$n$的排列逆序对数为$n!\times\frac{n(n-1)}{2}$\\
  \hline
  2017.10.11 & 传染病控制 & NOIP & AC & 树上按深度搜索\\
  \hline
  2017.10.11 & 斗地主 & NOIP & AC & 贪心+搜索\\
  \hline
  2017.10.12 & lost & test & AC & 树上维护凸包,二分弹栈\\
  \hline
  2017.10.12 & knows & test & AC & 线段树维护极长子序列\\
  \hline
  2017.10.12 & starway & test & AC & Prim最小生成树\\
  \hline
  2017.10.13 & 解方程 & NOIP & AC & 取模乱搞\\
  \hline
  2017.10.13 & 飞扬的小鸟 & NOIP & AC & 恶心的DP,类似背包\\
  \hline
  2017.10.15 & 火柴排队 & NOIP & AC & 贪心,树状数组求逆序对\\
  \hline
  2017.10.15 & 转圈游戏 & NOIP & AC & 水题,直接快速幂模拟\\
  \hline
  2017.10.15 & CF Round Div2 & CF  & AC A,B,C & 前三题以结论为主,比较简单,后面两题不会做.\\
  \hline
  2017.10.15 & 母亲的牛奶 & USACO & AC & 暴力搜索\\
  \hline
  2017.10.15 & Mashmokh & CF414B & AC & DP计数问题,$f_{i,j} = \sum_{k|j}{f_{i-1,k}}$\\
  \hline
  2017.10.16 & sequence & test & AC & 莫队 + two-pointers预处理\\
  \hline
  2017.10.16 & graph & test & AC & 邻接矩阵乘法,或倍增floyd,这个考场上不会\\
  \hline
  2017.10.16 & work & test & AC & 11:00才看懂题,直接模拟就好了\\
  \hline
  2017.10.16 & CF Round Div2 & CF & AC A,B,C,D & 题面复杂,主要是以推结论为主\\
  \hline
  2017.10.17 & slope & test & AC & 判断一下相邻两点的斜率就可以了,注意负数\\
  \hline
  2017.10.17 & segment & test & AC & 加法贡献的概率为$\frac{1}{k+1}$,$k$表示赋值个数,赋值操作为他们的平均数\\
  \hline
  2017.10.17 & path & test & AC & floyd问题的变形,没往这方向去想,边更新距离边更新答案\\
  \hline
  2017.10.17 & Bath Queue & CF28C & AC & 期望DP,考虑$f_{i,j,k}$表示$i$个房间,$j$个人,$k$个长度,考虑$i$的长度是否为$k$\\
  \hline
  2017.10.18 & sum & test & AC & 套路题,考虑单点贡献就可以了,但是要加快速乘\\
  \hline
  2017.10.18 & shopping & test & AC & 用线段树直接维护就好了,但是很难调,考场上没调出来,考后还调了一下午\\
  \hline
  2017.10.18 & road & test & AC & 分别考虑树上和环上边的贡献,DP就可以了\\
  \hline
  2017.10.19 & DZY Love Colors & CF444C & AC & 线段树维护相同颜色区间,均摊分析\\
  \hline
  2017.10.19 & 联合权值 & NOIP & AC & 考虑距离为2的点就是兄弟,对每个点算一遍就可以了\\
  \hline
  2017.10.19 & Elephant & CF258D & AC & 概率DP\\
  \hline
  2017.10.19 & Coin & USACO & AC & 博弈论,倒过来DP,优化考虑DP重叠的部分,有点像完全背包的思想\\
  \hline
  2017.10.21 & graph & test & AC & 贪心构造题,考场上好像连题目都看错了,改一下输出顺序可以加40pts\\
  \hline
  2017.10.21 & permutation & test & AC & 比较神的思路,转成拓扑排序,考场上忘了改格式丢了50pts\\
  \hline
  2017.10.21 & tree & test & AC & 结论题,答案就是权值之和,没注意这个结论,只写了一个60pts的暴力求稳\\
  \hline
  2017.10.22 & 天天爱跑步 & NOIP & AC & 线段树合并,维护向上,向下的路径信息,减去LCA上的就可以了\\
  \hline
  2017.10.22 & 子串 & NOIP & AC & DP计数问题,加个滚动优化空间\\
  \hline
  2017.10.22 & Wie & POI2009 & AC & 状压DP+最短路\\
  \hline
  2017.10.23 & 电话线 & USACO & AC & 二分答案+最短路\\
  \hline
  2017.10.23 & 运输计划 & NOIP & AC &二分+树上差分\\
  \hline
  2017.10.23 & Why Cow Cross Road & USACO & AC & 贪心\\
  \hline
  2017.10.23 & How many Trees & CF9D & AC & 计数DP\\
  \hline
  2017.10.24 & strip & CF488D & AC & two-pointers + 线段树维护DP\\
  \hline
  2017.10.24 & 灾后重建 & Luogu1119 & AC & Floyd变形\\
  \hline
  2017.10.24 & Royal & CF875F & AC & 图论,最大基环树\\
  \hline
  2017.10.24 & 多项式乘法 & UOJ34 & AC & FFT\\
  \hline
  2017.10.29 & 吃鸡 & test & AC & 树形DP,考虑$i$的子树内怎么走的\\
  \hline
  2017.11.1 & color & test & AC & 带修改莫队可以过,考场上脑补了一个块的大小不对\\
  \hline
  2017.11.1 & pipe & test & AC & 维护最小生成树即可,考场上太急了没写完\\
  \hline
  
\end{longtable}

\end{document}
