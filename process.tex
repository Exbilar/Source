\documentclass[landscape]{article}
\usepackage{ctex}
\usepackage{amsmath}
\usepackage{amsfonts}
\usepackage{amssymb}
\usepackage{graphicx}
\usepackage{colortbl}
\usepackage{fancyvrb}
\usepackage{longtable}
\usepackage{xcolor}
\usepackage[hidelinks]{hyperref}
\usepackage[affil-it]{authblk}
\usepackage[top = 1.0in, bottom = 1.0in, left = 1.0in, right = 1.0in]{geometry}
\usepackage{amsthm}

\newcommand\spc{\vspace{6pt}}
\newcommand{\floor}[1]{\lfloor {#1} \rfloor}
\newcommand{\ceil}[1]{\lceil {#1} \rceil}
\newcommand*\chem[1]{\ensuremath{\mathrm{#1}}}

\newtheorem{theorem}{Theorem}[section]
\newtheorem{lemma}[theorem]{Lemma}

\date{Latest Update : \today}
\title{Summer Process table}
\author{$\mathcal Exbilar$}

\begin{document}

\maketitle

\begin{longtable}{ccccccccccc}
  
  \hline
  Date & Name & Source & Status & Algorithm\\
  \hline
  2017.6.23 & K大数查询 & ZJOI2013 & AC & 线段树套权值线段树\\
  \hline
  2017.6.24 & 排序 & HEOI2016 & AC & 线段树 + 二分答案\\
  \hline
  2017.6.25 & KIN & POI2015 & AC & 线段树 + 链表\\
  \hline
  2017.7.1 & 小A的糖果 & 洛谷月赛 & AC & 贪心水题\\
  \hline
  2017.7.1 & 松江1843路 & 洛谷月赛 & AC & 自己写了三分,好像正解是中位数\\
  \hline
  2017.7.2 & SAM-Toy Cars & POI2005 & AC & 贪心 + 链表\\
  \hline
  2017.7.2 & Sasha and Arrary & CodeForces 718C & AC & 线段树维护矩阵乘法\\
  \hline
  2017.7.2 & The Child and Sequence & CodeForces 438D & AC & 势能线段树区间取模\\
  \hline
  2017.7.2 & 线段树基础练习 & Uoj228 & AC & 势能线段树区间开根\\
  \hline
  2017.7.2 & 相逢是问候 & SHOI2017 & AC & 势能线段树+欧拉定理,现在还不是太明白那个欧拉的原理,留坑\\
  \hline
  2017.7.2 & 上帝与集合的正确用法 & Bzoj3884 & AC & 欧拉定理,虽然加深了理解,但还是不太懂上一题的某个操作\\
  \hline
  
\end{longtable}

\end{document}
