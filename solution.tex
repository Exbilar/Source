\documentclass[landscape]{article}
\usepackage{ctex}
\usepackage{amsmath}
\usepackage{amsfonts}
\usepackage{amssymb}
\usepackage{graphicx}
\usepackage{colortbl}
\usepackage{fancyvrb}
\usepackage{longtable}
\usepackage{xcolor}
\usepackage[hidelinks]{hyperref}
\usepackage[affil-it]{authblk}
\usepackage[top = 1.0in, bottom = 1.0in, left = 1.0in, right = 1.0in]{geometry}
\usepackage{amsthm}

\newcommand\spc{\vspace{6pt}}
\newcommand{\floor}[1]{\lfloor {#1} \rfloor}
\newcommand{\ceil}[1]{\lceil {#1} \rceil}
\newcommand*\chem[1]{\ensuremath{\mathrm{#1}}}

\newtheorem{theorem}{Theorem}[section]
\newtheorem{lemma}[theorem]{Lemma}

\date{}
\title{杂题题解}
\author{$\mathcal Exbilar$}

\begin{document}

\maketitle

\section{AGC014 A. Cookie Exchanges}

\subsection{题目大意}

起初有$A$,$B$,$C$,三堆石头,每一次操作将A堆石头的数量各分一半到$B$,$C$堆,对于$B$,$C$,堆也是这样,求操作多少次后其中有一堆的石头数量为奇数. $A,B,C \le { 10 }^{ 9 }$

\subsection{题解}

其实直接模拟就可以了,再考虑一下特殊情况就可以AC.这里简单说一下证明.
如果$A=B=C$,那么答案是无解或是0.
那么对于其他情况,我们考虑两堆石头数量的最值.不妨设 $A \ge B \ge C$,那么经过一次操作之后,石头数量分别变为:$\frac { B + C }{2}$,$\frac { A + C }{2}$,$\frac { A + B }{2}$,很显然的发现以前的最值只差由$A-C$变为了$\frac {A - C}{2}$,所以最多$O(\log{N})$次就能使三个数相同,也就是最多操作这么多次,所以时间复杂度是$O({\log {N}})$ 

\newpage

\section{AGC014 B. Unplanned Queries}

\subsection{题目大意}

在一棵由$N$个节点树上,初始每条边权值为$0$,给你$M$个操作,每次操作是将$u$,$v$路径上的所有边权加一,已知操作后每条边的权值为偶数,求是否存在这样的树满足条件. $N,M \le {10}^{5}$

\subsection{题解}

又是一道结论题.判断操作中每个点出现的次数奇偶性,如果每个节点出现的次数都是偶数,那么存在,否则不存在.感性的认知一下,边权的操作可以转化成点权,如果每个节点在操作中出现的次数都是偶数,那么不难构造出一棵树满足条件,我们可以只把操作中的$(u,v)$直接看成边,注意一下不要连成环,最后把无关节点全部挂在一个节点上就可以了.若有一些点$v$,它们的出现次数是奇数,考虑深度最大的一个点,它与它父亲所连成的边一定只被覆盖了奇数次.因为它不可能再向下扩展(深度最深),那么它与他父亲连的那条边就存在于以它为端点的所有操作中(YY一下).

\newpage

\section{CodeForces 366C}

\subsection{题目大意}

给你$n$个物品,每个物品有两个权值$A$和$B$,求一个方案,使得所选物品的$A$之和恰好为所选物品的$B$之和的$k$倍,且$A$之和最大.

\subsection{题解}

我们考虑转化一下做背包,将每个物品的花费变成$A-kB$,价值为$A$,那么一个花费为$0$的背包就是答案. 

\end{document}
